\documentclass{IEEEtran}
%% BIbliography Setup - Biblatex please
\usepackage[
backend=biber,
style=ieee,
]{biblatex}
\addbibresource{DBT_References.bib}

%% Packages
\usepackage{caption}
\usepackage [none] {hyphenat} 
\usepackage{siunitx}
\usepackage{graphicx}
\usepackage{amsmath}
\usepackage{subcaption}
\usepackage{pdfpages}
\usepackage{booktabs}

\markboth{First National Conference On Advanced Materials and their Applications , October 18 and 19 th 2023 Tipaza}{Last Name \MakeLowercase{\textit{et al.}}: Title}

\title{CDT 22 - Design, Build and Test. Sequential Instabilities for Actuating Aerodynamic Surfaces}
\author{First Author$^1$, Second Author$^2$, and Third Author$^3$\\
	$^1$Department of Electrical Engineering, University of X, X City, X Country\\
	$^2$Department of Computer Science, University of Y, Y City, Y Country\\
	$^3$Department of Mechanical Engineering, University of Z, Z City, Z Country\\
	\{first.author, second.author, third.author\}@cu-tipaza.dz}
\begin{document}
	\maketitle
	
	\begin{abstract}
		
	\end{abstract}
	
	\section{Introduction}
		Resilience to gust loading is a major consideration for aircraft design, yet extreme gusts, which can cause significant damage to airframes and aircraft systems, occur at low frequency \cite{Wu2019}. While rare, it is still necessary to design for these worst case events, resulting in a conservative structural design philosophy which adds mass and thus reduces fuel efficiency \cite{Alam2015}. Conventional wing spoilers can be used to alleviate aerodynamic loads during a gust, thereby reducing airframe stresses. However, such control surfaces utilise conventional, active actuation mechanisms such as hydraulic or electro-mechanical actuators. These mechanisms introduce significant mass and complexity to aircraft which results in a decrease to fuel efficiency and an increase in cost and manufacture time. In order to achieve reductions in aircraft mass, whilst still meeting design requirements for extreme gust loads, new mechanisms for control surfaces that lightweight and still able to effectively alleviate gust loads must be developed.
		
		

	\section{Methodology}
	
	
	\subsection{Equations}
	Here is an example of an equation:
	\begin{equation}
		f(x) = x^2 + 2x + 1
	\end{equation}
	
	\subsection{Tables}
	Here is an example of a table:
	\begin{table}[htbp]
		\centering
		\caption{Example Table}
		\label{tab:example}
		\begin{tabular}{|c|c|c|}
			\hline
			\textbf{Column 1} & \textbf{Column 2} & \textbf{Column 3} \\
			\hline
			Row 1, Column 1 & Row 1, Column 2 & Row 1, Column 3 \\
			\hline
			Row 2, Column 1 & Row 2, Column 2 & Row 2, Column 3 \\
			\hline
			Row 3, Column 1 & Row 3, Column 2 & Row 3, Column 3 \\
			\hline
		\end{tabular}
	\end{table}
	
	\subsection{Figures}
	Here is an example of a figure:
	\begin{figure}[htbp]
		\centering
		\includegraphics[width=0.4\textwidth]{example-image-a}
		\caption{Example Figure}
		\label{fig:example}
	\end{figure}
	
	\section{Results}
	
	\section{Conclusion}
	
	
    \printbibliography
	
\end{document}
